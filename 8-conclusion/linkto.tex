\subsection{Link to thesis}
This chapter presents two case studies exploring human-AI co-creativity in new media practice through public art installations. The first was commissioned by the ANU School of Cybernetics for their exhibition "Australian Cybernetics: a point through time" (presented at EvoMUSART 2022). The second was commissioned by the Sydney Opera House for their 50th anniversary celebrations (presented at SMC 2024).

Both installations, developed in collaboration with Uncanny Valley music studio, explored a novel co-creative role for generative AI: using an LLM as a semantic interpreter of environmental data (weather, building activity, economic indicators) to generate evolving audiovisual soundscapes through a music engine. This represents a new form of data sonification that goes beyond direct parameter mapping to semantically rich interpretation.

The core questions guiding this enquiry were: can generative technologies enable new creative practices by assuming novel roles? How well can they fulfil these roles and what challenges emerge? How does this inform my research questions regarding interaction design for effective co-creativity, human agency, and dialogic interaction?