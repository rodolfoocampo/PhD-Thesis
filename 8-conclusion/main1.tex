\chapter{Conclusion}\label{c:conclusion}

- In this conclusion I will colate the main findings and answer the research questions
- As a brief summary for the reader, in this thesis, I set out to investigate how to enable co-creativity between humans and AI. 
- In particular, how to design effective co-creative AI systems that mantain user involvement and creativity while leveraging the creative potential of artificial intelligence
- As specific research questions, I defined: 

\begin{enumerate}
    \item \textit{R3: What are the design principles can guide the development of effective generative AI systems that are co-creative with people?}
    \item \textit{R1: What is the potential of modelling dialogue in interaction design to enable effective human–AI co-creativity?}
    \item \textit{R2: What roles can humans and AI assume in co-creative processes?}
\end{enumerate}

- I investigated this through practice based research, literature reviews and user studies using prototypes. 
- The period in which this research was carried out was one of rapid changes
- Artificial intelligence exploded, and as such, it was a challenge to keep up
- However, I also recognised increasingly, the relevance of this research
- Artificial intelligence began being integrated in multiple areas
- In particular, arguably the most affected area and one with the most pressing debates are creative industries
- Given the generative Ai capabilities of AI

- I found that as such, the role of my research is to provide practice based research as a creative practioner and from the perspective of interaction design to help design tools that enhance the creativity of people rather than replace them
- As Koomen has pointed out, a lot of how AI is being integrated is as a horseless carriage, that is, designing with previous interaction modalities that limit the effectiveness of AI, and in some cases is detrimental to users and society at large. 

- Moreover, as Scheiderman has pointed out, new computing paradigms have historically required new modes of interaction with those. 
- We are at a point in history in which the development of generative AI represents a radical shift in a computing paradigm, such that it requires understanding how to interact with, effectively. 

As such, this was the first of my research questions:
To date, no research has published design principles for generative AI that is co-creative. This is the main contribution of my thesis, and the design principles are outlined as follows, based on the learnings from the research conducted in the previous chapters. 

\section{Design Principles for Co-creative generative AI systems}

- Principle 1: design electric bicycles, not self driving cars
    -  Control was a key theme throughout the thesis
    -  In the co-creative writing experiment, participants often discussed the need to iterate on writing
- Principle 2: human as gardener
- Principle 3: 

Principle 1: Design electric bicycles, not self-driving cars.
Principle 2: Co-creativity thrives in conversation, not command.
Principle 3: Give human and AI a shared canvas, not just a chatbox.
(Still under discussion - focused on AI visibility & mutual understanding)
Previously: "An honest AI shows its seams, revealing both power and limits."
Matter-of-fact alternative we discussed: "Facilitate mutual understanding by making AI's functions visible and its interpretation of human input accurate."
Principle 5: Let AI open new doors, not just automate old paths.
(Still under discussion - focused on iteration)
Previously: "Iteration is creation's heartbeat; AI must learn its rhythm." (User found this "corny")
Matter-of-fact alternative we discussed: "Design for iterative refinement, as creation is a process of progressive development."
Principle 7: A co-creative AI knows where it is, and where it's been.
Principle 8: Seek the alien in the algorithm; true co-creativity is more than mimicry.
Principle 9: In the orchestra of AI, the human is the conductor, shaping a symphony from many voices.
Principle 10: Provide expressive interfaces beyond text.