In this freewriting session I want to get some thoughts out regarding how to structure the conclusions of this thesis. Throughout this thesis, I explored various studies and practice based research use cases in which I explored the design and practice of human-AI co-creativity. 

In particular, my intention was to both conduct studies with users and also engage in my own practice in which I examine my particular experince to help illuminate how to succesfully design systems that are co-creative with humans. 

The main contribution of this work is providing design principles that help designing of cocreative systems create systems that can be cocreative with humans. 

The design principles are as follows: 

1. Create electric bicycles, not self driving cars
2. Design for the human as a gardener
3. Open new doors, rather that automating old pathways. 
4. Expand the palette of creative possibilities. 

Create electric bicycles, not self driving cars. 

This particular principle encompasses a lot of what I learned throughout my research. 
It has several dimensions worth exploring. 
In the first place, it encapsultates the idea that humans want to remain at the wheel: they want to mantain control and initiative. 
In my study in Chapter 4, a common theme was that people wanted to drive the story. 
However, a common limitation they expressed is that the AI did everything and that their voices were lost. 
In the emerging literature that emerges throughout the time of this research, this has become a common theme. 
Generative AI systems that limit user's agency by autoomating and limiting their control. 
However, control is difficult. 
As I discussed in chapter 5, in the use case for the production of a magazine issue, creative control was one of the most important elements for the professional use o these systems.
For example, controlling the style, or controlling the composition and controlling the expression and so on. 

This problem encompasses both technical challenges at the model level, and interaction design challenges. It is not possible to draw a clear separating line between these two. How the technical challenges are solved often involve novel interactions. 
Moreover, control means different things in different settings. 
In the case of image generation in may mean style, and mean composition. In the case of text it may mean the voice, but also the plot, the wording, the length. In the case of music, it may mean the style but also cintrolling things like tempo, key, instrumentation, etc. 
However one key commonality is that text is a limited form of interaction, and does not provide a sufficient level of control. 
This presents a tension and a paradox. 
The rise of generative artificial intelligence, in terms of adoption, has largely been driven by the rise in text-to-X, where users can prompt models to perform certain things with the simplicity of natural language. 
As a result, we currently are seeing that interfaces have converged towards prompting modalities.
So this is limited, first, because this limits control. For example, users often express it is difficult to express a visual ide in text, and designers onteracting with image generation tools have said they are visual, not verbal thinkers. In the same way, a musician may find it hard to express the nuance of the music they intend simply via language. 
Therefore, a core opportunity lies in providing expressive interfaces beyond text that allow users to convey creative intent, and express and steer the model more naturally. 
Moreover, with this sort of multimodal interfaces, users can make direct contributions to the creation, rather than be relegated to the role of instructors. 
In text, prompt only modalities, users tend to simply sit back to the role of instructors, in which they are simply asking the model to perform things on their behalf, as was shown in my study in chapter 4. 
Provided interfaces that allow users to make direct contributions in the modality of the creation makes them be more involved and participate as co-creators rather than merely outsources. 

In the case of my co-writing study, this led to the proposal of providing both a space to interact about the creation and to interact through the creation. Having a shared creative canvas where both can make contributions, and also having a separate space to communicate. In the case of music for example, this would involve sharing a space for collaboration where both can play music, in addition to having a chat window where they both can discuss the music. In the case of image generation, this can mean having a canvas where both can make direct contributions at rhe pixel level, drawing, designing elements etc. 

These interfaces can enable the user to have more control, and become more involve as a co-creator. 

But beyond the collaborative aspect of having an interface to interact both through and about, expressive interface can make the experience more fun and enjoyable, which is an important component of co-creation. 

In my case study in the installation for system of a sound, which can be understood as a network of collective crerativity. It was collective creativity, as I was on one hand, a co-creator with the model, but also involving the people as co-creators, as we provided an interface for them to participate in the making of the soundscape, using a gestural interface, which allowed them to control the playing of digital strings on the screen, whose sounds depended on the data that was being displayed. This gestural interface allowed user to become co-creators and participate in the co-creative experience in a way that was fun (reference the paper with Josh, for the building). 

Another playful interface was explored in Narrative Device, a extremely simple one, in which people provided two words, and simply hit generate and the system crafted a short story integrating those two concepts. While this interface was not intended to provide a granular level of control that perhaps a professional authors would require, it served as an epxloration of the potential of AI to inspire creativity by mixing disparate themes together. 

The interface proved incredibly successful and provided me with the insight that playful combinatorial interfaces can be an effective way to engage users. This interface did not provide control beyond simply providing the two themes. However, thats ok. It may the case that in some systems, given their particular scope or intent, this is what control looks like. Here the user and Ai did not share a creative canvas, and the interaction was not dialogic, but it can be argued there is a level of co-creativity in having the users think divergently in terms of what disparate concept to select, and the AI engaged creatively by weaving them into a story. In this case, limited control iin the form of a simple input introduces a constraint, which may aid creativity, as it has been shown in research on creativity. This may be a good thing. 

And this may highlight another of the challenges of prompt based interfaces. They are open ended, and as such, they provide the illusion that the system can do everything. Often these tools are sold: do whatever you can imagine. However, in reality, users often find themselves frustrated when they do ask for aything and the system fails. Constraining the input and being clear about what the system does, like in the case of narrative device, where it said: it just generate a story based on your topics. In this case, the capability of the system was visible, and the interface was constrained. 


Now that we are on the topic of collective creativity, its worth discussing this aspect of co-creativity. Understanding creativity as a collective pursuit is a core theme and conceptual assumption of my research. The system of a sound case study sought to make this explicity, by creating an installation that produced a creative output in the form of a soundscape, by engaging an AI, that interacted with its environment, and responsed to it, where the public was also involved in the co-creation, and of course, us as human creators of the installation involved at the meta creative level. 

I continud this theme in the installation performed for the opera house, in which we intended to bring the building as a co-creator as well. While we did no provide an interface for people to directly contribute to the soundscapce being generated, their presence themselves affected the co-creation, as the installation responded to changes in energy use, temperature of the building and so on, and these were all influenced by the amount of people in the building. 

My installation done for the Milan Design Week, with Studio Snoop, also presented similar ideas of collective creativity. In this case, the network of creativity involved was me as the creator of the system, the system itself which helped ideate designs, the designers who interacted with this system to ideate designs, and the participants and public that interacted with this system providing feedback, which was used to iterate the designs every day over the course of a week. In this case, considering a lens of creativity as collective, allowed me to design a system that brought in the audience and participants into an iterative process of interaction. The AI was used not only as an ideation partner to craft the designs which were exhbited, but it utilised its unique capabilities of being able to conversationally interact with many people in parallel, collect information, and then summarise this information for designers to iterate on the designs. This is a unique capability of AI, an illustrates the potential of using AI not simply to automate things that humans already to, but to expand possibilities, and allow teams of creatives to do things that were otherwise not possible. 

Same as with the opera house and system of a sound, we engaged the AI not to automate something that humans already do, but instead to enable a new form of creative production not existing at the moment: generating a soundscape, 24/7, based on the interpretation of a stream of complex data, in which the data was semantically interpreted and mapped to musical qualities, rather than mapped linearly as many sonification approaches do. 

Creating systems that expand the lens from that of co-creativity as an individual pursior by human and machine, into a wider collective, involving collectives of people, as well as environments can help design co-cretaive systems that are nore contextually aware, interesting, and perhaps even more ethical, considering how the creation sits within a larger context that affects and responds to other people. 

With this, I have several threads for design principles: 

- Provide control through rich modalities beyond text
- Provide a share space to co-create int he modality of the creation
- Provide space to interact both through and about the creation
- Enable rich, expressive and playful interfaces
- Constrains enhance creativity
- Clearly communicate the limitations and capabilities of the system
   -- This can be done implicitely, through interaction design that makes the affordances visibile, like in the case of Narrative Device
  -- Or it can be done explicitely. 
  -- However, for example, in the writing study, I sought to provide this information verbally, simply by having text that communicates what the system could do and how. However, some users expressed confusion about what the system was actually capable of, and expressed desire for having an interface that had simple buttons that coul trigger actions. So, instead of seeking to provide this visibility by expressing it verablly, the interface can take a "show don't tell approach" the interface itself communicates what is possible. For example, narrative device communicates it can mix two concepts simply because this is the only thing the interface allows you to do. In the case of a co-writing iterface, the system could have button to trigger specific actions: like refining and editing a text, conitnuing the text, proviuding feedback on it, generating ideas and so. 
-- Text may be limited because it's open ended and doesn't clarify what are the real limitations and capabilities. 


- Creativity is collective
- Use AI to expand the palette of creative possiblities, rather than to automate the possibilities of the human
- Creative processes are iterative. Design for iteration. 
