\chapter{Abstract}

Progress in computational generative intelligence brings exciting opportunities and urgent questions. This thesis explores the possibility of human-machine co-creativity, where the distinct creativities of people and machines are blended to produce new and interesting outcomes.

While generative models are increasingly powerful, their usability hinges on yet unanswered questions at the interaction layer. Crucially: how can we design effective interactions between humans and computational intelligence that maintain human agency, while allowing for mutually enhanced creativity? Addressing this is the core aim of my thesis.

New computational paradigms have required the development of new interaction paradigms. In consequence, numerous interaction design principles exist in the field of human-computer interaction. However, no such principles exist yet for human-AI co-creativity.

This thesis combines an analysis of emerging literature with original research to fill this gap. In particular, I examine three research questions. First, I examine how interaction design can influence the roles humans and AI play in creative activities. Secondly, I explore how to enable dialogues between humans and machines that yield mutual influence and understanding to better support co-creativity. Lastly, I investigate interaction design principles that can guide the development of effective human-AI co-creativity that mantains human agency. 

I employ a mixed-methods practice-based approach leveraging interaction design research and my own creative practice. First, I perform an analysis of the emerging literature. Then, in chapter 3, I develop dialogic interaction as an interaction design concept, arguing that successful creative dialogues require iterative interaction both \textit{through} and \textit{about} the creation. In chapter 4, I develop a co-creative system for writing that implements a hybrid interface to better support this interaction both \textit{through} and \textit{about} the artifact. Across two user studies, I show this interface leads to greater user involvement and perceived agency, compared to chat-only interfaces. 

In chapter 5, I describe a case study with the Australian Financial Review, exploring the potential for image-based generative artificial intelligence to support real-world creative workflows. I identify three critical challenges for the adoption of these tools: a lack of consistency, limited granular control, and inadequate support for iterative workflows. Lastly, in chapter 6, I describe the creation of two generative public art installations, one commissioned by the Australian National University and the other one commissioned by the Sydney Opera House. These case studies discuss the potential and challenges for AI to assume novel roles not currently assumed by humans, thus enabling new creative possibilities and offering an alternative beyond a future of creative automation and role displacement.

In the final chapter, I argue that transcending the current linear prompt-based and purely conversational interaction paradigms is crucial for co-creativity. Such interactions lead to users primarily assuming roles at the "intentional space" of goals and ideas, while the AI primarily assumes creative execution roles at the “action space". This leads to what I describe as \textit{severed creative agency}: a disconnect between creative intention and action, which can lead to reduced user involvement, erosion of skills, and reduced intrinsic enjoyment in creative processes.

As the core contribution, I provide a set of 11 actionable design principles for creating effective co-creative AI systems, derived from the dialogic framework established in chapter 3, informed by the literature review and subsequent original research. They are organised around the dimensions of Iteration, Communication, Collaboration, and Integration. 

It is my hope that this thesis is useful for interaction designers, developers, researchers and organisations, such that they can develop co-creative systems with confidence. It is also my hope that it is useful for artists and practitioners, inspiring creative possibilities at the intersections with other intelligences.