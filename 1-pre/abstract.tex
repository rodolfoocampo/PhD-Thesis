\chapter{Abstract}

Recent progress in generative artificial intelligence opens the possibility of human-machine co-creativity. However, realising this largely hinges on yet unanswered questions at the interaction layer. Crucially: how can interactions between humans and computational intelligence be designed so that their unique forms of creativity are mutually enhanced and blended, while human agency is maintained? Addressing this constitutes the core aim of my thesis.

While numerous interaction design principles exist in the field of human-computer interaction, no such principles exist yet for human-AI co-creativity. This thesis combines an analysis of emerging literature with original interaction design and creative practice-led research to fill this gap. 

Co-creativity requires mutual influence and understanding. However, current linear and request-based human-computer interaction design paradigms are limited in affording this. I argue that modelling dialogue is a promising approach. Consequently, I developed a theoretical framework for dialogic co-creativity where human and computer iteratively interact \textit{through} and \textit{about} the creation.  I then used this framework as a navigational lens across three main pieces of research. 

First, in Chapter 4, I developed a dialogic co-creative writing system supporting interaction \textit{through} and \textit{about} the writing by integrating a collaborative text editor and a conversational space. Across two user studies, I show that this interface leads to greater user involvement and perceived agency, compared to chat-only interfaces.  

In Chapter 5, I present a case study conducted in collaboration with a nationally printed Australian magazine, exploring the potential and challenges of image-based generative AI in real-world creative workflows through the co-production of one of their issues. Three main challenges were identified: lack of generative consistency, limited stylistic and structural control, and inadequate support for iterative workflows. 

Lastly, in Chapter 6, I describe the creation of two generative public art installations leveraging generative language models for real-time audiovisual performance. Generative artificial intelligence opens up new creative possibilities in new media practice by assuming novel co-creative. However, they remain difficult to steer towards human creative intent. 

In the final chapter, I argue that the dominant interaction design paradigms of prompting and conversational request lead users to primarily assume roles in the 'intentional space' of goals and ideas, while AI assumes creative execution roles in the 'action space'. As a result, users become increasingly detached from the creative process, leading to what I describe as \textit{severed creative agency}: a disconnect between creative intention and action. As a result of this disconnect, users struggle to materialise ideas into outputs, and become less involved at the creation level, eroding skills and reducing their sense of agency, ownership and enjoyment. However, generative AI has the potential to enable new creative operations by collaboratively enhancing human creativity. Balancing this is what I describe as \textit{the core tension in human-AI co-creativity}.

As the main contribution, I provide a set of 11 actionable design principles for creating more effective co-creative AI systems, derived from the dialogic co-creativity framework and focused on enabling mutual understanding, adaptation, enhanced user involvement and human creative agency. 

This thesis contributes an understanding for how to develop co-creative systems with confidence. It is also my hope that it is useful for artists and practitioners, inspiring creative possibilities at the intersection with other intelligences.