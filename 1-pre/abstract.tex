\chapter{Abstract}

Recent progress in generative artificial intelligence (AI) opens the possibility of human-machine co-creativity. However, realising this possibility depends on answering a key question at the interaction layer. How can we design interactions between humans and AI so that their unique forms of creativity are mutually enhanced and blended, while human agency is maintained? Addressing this question is the core aim of my thesis. 

Numerous interaction design principles exist in the field of human-computer interaction, yet no such principles exist for human-AI co-creativity. This thesis combines an analysis of emerging literature with original interaction design and creative practice-led research to fill this gap. 

Co-creativity requires mutual influence and understanding. However, current linear and request-based human-computer interaction design paradigms are limited in affording such influence and understanding. I argue that modelling dialogue is a promising approach. Consequently, I develop a theoretical framework for dialogic co-creativity where human and computer iteratively interact \textit{through} and \textit{about} the creation. I then use this framework as a navigational lens across three main pieces of research.

First, in Chapter 4, I develop a dialogic co-creative writing system supporting human-AI interaction \textit{through} and \textit{about} the writing process by integrating a collaborative text editor and a conversational space. Across two user studies, I show that this interface leads to greater user involvement and perceived agency, compared to chat-only interfaces.

In Chapter 5, I present a case study conducted in collaboration with a nationally printed Australian magazine, exploring the potential and challenges of image-based generative AI in real-world creative workflows through the co-production of one of their issues. I identify three main challenges: a lack of generative consistency, limited stylistic and structural control, and inadequate support for iterative workflows.

Lastly, in Chapter 6, I describe the creation of two generative public art installations leveraging generative language models for real-time audiovisual performance. Generative AI opens up new creative possibilities in new media practice by assuming novel co-creative roles. However, generative AI systems remain difficult to steer towards human creative intent.

In the final chapter, I argue that the currently dominant interaction design paradigms of linear prompting and purely conversational interfaces lead users to primarily assume roles in the `intentional space' of goals and ideas, while AI assumes creative execution roles in the `action space'. As a result, users become increasingly detached from the creative process, leading to what I describe as \textit{severed creative agency}: a disconnect between creative intention and action. 

I then provide five design principles to address this, which are derived from a synthesis of the research: \textit{serendipity, personalisation, expressiveness, iteration and collaboration}. These principles and their corresponding specific guidelines for implementation, can be understood as strategies for enabling mutual understanding, adaptation and co-creative dialogues between humans and AI. 

This thesis contributes to our understanding of how to develop co-creative systems with confidence. It is my hope that it also inspires artists and practitioners to explore the creative possibilities at the intersection with other intelligences.
