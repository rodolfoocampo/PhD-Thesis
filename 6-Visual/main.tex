\chapter{Integrating Generative AI into Creative Workflows: Dealing with Consistency, Scene Control, and Refinement in a Professional Image Generation Case Study} \label{c:tc6} 

\section{Link to thesis}

This case study shifts the focus to the domain of image-based generative AI, exploring research questions through a practical case involving real-world creative production. I collaborated with the Australian Financial Review to produce visuals for their 2023 Power Issue, which features who they deem to be the most powerful people in Australia. The aim was to produce "impossible photography" of the subjects to tell stories about their personality and work difficult or impossible with traditional photography. 

On one hand, this exploration contributes to understanding effective human-AI co-creativity by identifying how generative tools can enable new creative possibilities and use cases. Secondly, a core objective of this case study was to examine the limitations of current tools in supporting real-world creative production. 

Findings from this work primarily inform my research question 3, investigating design principles for human-AI co-creativity, by highlighting crucial challenges. A significant limitation identified is the difficulty in enabling precise control over style and structure particularly when relying solely on prompting interfaces. A second challenge was iterative exploration and refinement of promising images to pursue interesting directions. Exploring and further refining is a crucial dynamic in creative processes, and remains a critical challenge for successful human-AI co-creativity.

Provided below is the publication covering the case study, presented as a short paper and poster at the 2024 International Conference on Computational Creativity (ICCC):

\textbf{Publication in lieu of Chapter:}


Integrating Generative AI into Creative Workflows: Dealing with Consistency, Scene Control, and Refinement in a Professional Image Generation Case Study. Ocampo Blanco, Rodolfo, \& Bown, Oliver. (2024). In \textit{Proceedings of the International Conference on Computational Creativity (ICCC)}. Jönköping, Sweden. 

\includepdf[pages=-]{7-techchapter/ICCC_short.pdf}

