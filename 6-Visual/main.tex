\chapter[Human-AI Co-Creativity in Visual Production]{Integrating Generative AI into Creative Workflows: Dealing with Consistency, Scene Control, and Refinement in a Professional Image Generation Case Study
} \label{c:tc6} 

This case study shifts the focus to the domain of image-based generative AI, exploring research questions through a practical case involving real-world creative production. I collaborated with the Australian Financial Review to produce visuals for their annual Power Issue which features “the most powerful people in Australia”. The aim was to produce “impossible photography” of the subjects, to tell stories about their personality and work in new ways, otherwise impossible with traditional photography. 

With this, on one hand, this exploration contributes to understanding effective human-AI co-creativity by identifying how generative tools can enable new creative possibilities and use cases.

Secondly, a core objective of this case study was to examine the limitations of current tools in supporting real-world creative production. 

Findings from this work primarily inform my interaction design principles (R3) by highlighting crucial challenges. A significant limitation identified is the difficulty in enabling precise control over style and structure particularly when relying solely on prompting interfaces. A second notable challenge was iteratively refining promising images to pursue interesting directions. Exploring and further refining is a crucial dynamic in creative processes, and remain a critical challenge for successful human-AI co-creativity. 

This case study further illuminates ethical considerations. While intention was partly to spark a debate around deep fakes, the discussion in this paper could itself help produce deep fakes. Moreover, tools that better integrate into creative production as argued here could lead to job displacements. Even if the intention was to explore possibilities not afforded by cameras, there is little preventing commercial usage to simply produce photographs passing as real: threatening the employment of models, actors, photographers and other occupations. 


\includepdf[pages=-]{7-techchapter/ICCC_short.pdf}

