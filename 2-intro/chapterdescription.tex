\chapter{Abstract}

Progress in computational generative intelligence brings exciting opportunities and urgent questions. This thesis explores the possibility of human-machine co-creativity, where the distinct creativities of people and machines are blended to produce new and interesting outcomes.

While generative models are increasingly powerful, their usability hinges on yet unanswered questions at the interaction layer. Crucially: how can we design effective interactions between humans and computational intelligence that maintain human agency, while allowing for our mutual creativities to be blended and enhanced? Addressing this is the core aim of my thesis.

New computational paradigms have required the development of new interaction paradigms. In consequence, numerous interaction design principles exist in the field of human-computer interaction. However, no principles exist yet for human-AI co-creativity.

This thesis combines an analysis of emerging literature with original research to address three specific research questions:

\begin{itemize}
\item How does interaction design influence the roles humans and AI assume in creative processes?
\item How does modelling dialogic design help design effective human-AI co-creativity?
\item What design principles can guide the development of effective human-AI co-creativity?

\end{itemize} I examine these questions through a mixed-methods approach, primarily driven through practice-based research.

The research begins by outlining the challenges of human-AI co-creativity and reviewing the relevant literature to propose a dialogic framework for interaction design. This theoretical grounding is followed by a series of empirical investigations, including the development of a hybrid co-creative prototype, a real-world case study with the Australian Financial Review, and the creation of two generative soundscape installations for public art commissions. These studies explore how different interaction models impact user agency and open new possibilities for creative practice.

The findings indicate that transcending the current linear prompt-based and conversational based interaction paradigms is crucial. Such interactions lead to users primarily assuming roles at the "intentional space" of goals and ideas, while the AI primarily assumes creative execution roles at the “action space". This leads to what I describe as "severed creative agency": a disconnect between creative intention and action, which can lead to reduced user involvement, erosion of skills, and reduced the intrinsic enjoyment in creative processes.

I found that a hybrid interface enabling interaction both through and about the creation can lead to more balanced roles and increase user involvement, ownership, and agency.

The practice-led case studies identify three critical challenges in current generative AI tools, a lack of consistency in subject and object generation, limited granular control, and inadequate support for iterative refinement. Interaction design can address this by enabling interfaces that support guided exploration, expressive multimodal controls, output-input feedback loops, dialogic interfaces and prioritising training models for iterative rather than one-off generations. Furthermore, the research demonstrates the potential for AI to assume novel roles not currently assumed by humans, enabling new creative possibilities and offering an alternative beyond a future of creative automation and role displacement.

As the core contribution, I provide a set of 11 actionable design principles for creating effective co-creative AI systems, derived from the dialogic framework established in chapter 3, informed by the literature review and subsequent original research. They are organised around the dimensions of Iteration, Communication, Collaboration, and Integration. Ultimately, this work argues that by designing for dialogic interaction, where users and AI engage in close iterative loops both through and about the creative process, a better alignment between intention and action can be achieved, contributing to effective blending of creativity between humans and machines.

It is my hope that this thesis is useful for interaction designers, developers, researchers and organisations, such that they can develop co-creative systems with confidence. It is also my hope that it useful for artists and practitioners, inspiring creative possibilities at the intersections with other intelligences.